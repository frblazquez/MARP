\documentclass[12pt]{article}
\usepackage[margin=1in]{geometry} 
\usepackage{amsmath,amsthm,amssymb,amsfonts}

\newcommand{\N}{\mathbb{N}}
\newcommand{\Z}{\mathbb{Z}}
 
\newenvironment{problem}[2][Problema]{\begin{trivlist}
\item[\hskip \labelsep {\bfseries #1}\hskip \labelsep {\bfseries #2.}]}{\end{trivlist}}

\usepackage[utf8]{inputenc}                 % Using Unicode encoding
\usepackage[spanish]{babel}                 % Writing in Spanish language
%\usepackage{listings}                       % For including C++ code
\setlength{\parindent}{0mm}                 % Paragraph indentation null
\setlength{\parskip}{\baselineskip}         % Distance among paragraphs

%--------------------------------------------------------------------------
%   PARA INCLUIR CÓDIGO C++ EN EL DOCUMENTO
%--------------------------------------------------------------------------
\usepackage{color}
\definecolor{gray97}{gray}{.97}
\definecolor{gray75}{gray}{.75}
\definecolor{gray45}{gray}{.45}

\usepackage{listings}
\lstset{ frame=Ltb,
framerule=0pt,
aboveskip=0.5cm,
framextopmargin=3pt,
framexbottommargin=3pt,
framexleftmargin=0.4cm,
framesep=0pt,
rulesep=.4pt,
backgroundcolor=\color{gray97},
rulesepcolor=\color{black},
%
stringstyle=\ttfamily,
showstringspaces = false,
basicstyle=\small\ttfamily,
commentstyle=\color{gray45},
keywordstyle=\bfseries,
%
numbers=left,
numbersep=15pt,
numberstyle=\tiny,
numberfirstline = false,
breaklines=true,
}

% minimizar fragmentado de listados
\lstnewenvironment{listing}[1][]
{\lstset{#1}\pagebreak[0]}{\pagebreak[0]}

\lstdefinestyle{consola}
{basicstyle=\scriptsize\bf\ttfamily,
backgroundcolor=\color{gray75},
}

\lstdefinestyle{C}
{language=C,
}

\begin{document}

%---------------------------------------------------------------------------------------
% TÍTULO Y AUTOR
%---------------------------------------------------------------------------------------
\title{\textbf{Hoja 6 - Ejercicio 14}}
\author{Francisco Javier Blázquez Martínez}
\date{}
\maketitle
 
%---------------------------------------------------------------------------------------
% ENUNCIADO DEL PROBLEMA
%---------------------------------------------------------------------------------------
\begin{problem}{6.14}
Se dispone de dos vasijas con capacidades enteras distintas $C_1$ y $C_2$, dadas en 
litros, y de una fuente de agua ilimitada. Se desea alcanzar en una de ellas una 
cantidad exacta de litros de agua C, para lo cual podemos realizar acciones de llenado 
y vaciado de una vasija, o de trasvase de agua entre las dos vasijas, sin derramar nada. 
En todo trasvase de una vasija A a otra B, o bien B se llena completamente, o bien A se 
vacía completamente (o bien ambas cosas a la vez). Por ejemplo, si $C_1 = 7$ litros y 
$C_2 = 11$ litros, es posible conseguir en no más de 10 acciones la cantidad de 6 litros
en una de ellas. Se desea saber si el problema tiene solución en a lo sumo n acciones 
elementales.

Diseñar un algoritmo que, dados C, $C_1$ , $C_2$ y n arbitrarios, diga si el problema 
tiene solución, y en caso afirmativo proporcione una secuencia de acciones elementales 
para alcanzarla. Solo deben usarse las ideas de programación dinámica, sin recurrir a 
ninguna pericia que maneje la teoría de números.
\end{problem}

%---------------------------------------------------------------------------------------
% SOLUCIÓN
%---------------------------------------------------------------------------------------

\textbf{Solución:} \\

%\setlength{\parskip}{0mm}
%\setlength{\parskip}{\baselineskip}

\setlength{\parskip}{0mm}
Observaciones preliminares:

\begin{itemize}
    \item Consideramos acciones el llenado, vaciado o trasvase de agua de una vasija a 
          otra siempre de acuerdo con las condiciones del enunciado.
    \item El enunciado dice ``\textit{a lo sumo en n acciones elementales}'' pero en esta
          solución hemos optado por una recurrencia más rígida (posibles cantidades a 
          obtener en \textit{exactamente} n acciones elementales). Esto hace que haya que
          controlar la respuesta al problema planteado durante el llenado de tabla.
\end{itemize}


Planteamos a partir de aquí la siguiente recurrencia: \\ 
    
    posible($C_1$, $C_2$, C, 0) = \{(0,0)\}\\
    posible($C_1$, $C_2$, C, k) = una\_acción(posible($C_1$, $C_2$, C, k-1))\\
    
Donde el método ``una\_acción'' toma un conjunto de estados de llenado de $C_1$ y $C_2$ y
devuelve el conjunto de posibles estados de llenado finales partiendo de estos y ejecutando
una única acción. Aquí se destaca el segundo punto de las observaciones, siempre aplica 
una acción. 
\setlength{\parskip}{\baselineskip}

A partir de aquí, necesitaremos (debido a que nos piden también la reconstrucción de la 
solución en caso de que exista) un espacio en memoria de hasta $k \times C_1 \times C_2$.
Esto es porque vamos a trabajar sobre matrices de $C_1 \times C_2$ booleanos, una por 
cada acción a ejecutar de forma que si en la i-ésima matriz el elemento (a,b) es true es
porque podemos llegar a un estado en el que la vasija $C_1$ tiene a litros de agua y la
vasija $C_2$ tiene b litros de agua con exactamente i acciones.


Aclaraciones finales:
\setlength{\parskip}{0mm}

\begin{itemize}
    \item Sobre el llenado de las tablas: \\
          No hay mucho que explicar aquí. Partimos de la tabla del estado anterior y aplicamos
          todas las reglas del enunciado a cada estado asociado a una casilla con valor true.
    \item Sobre la condición de parada: \\
          Si en algún momento, durante el llenado de tablas, marcamos una casilla cualquiera de
          la columna o fila C-ésima como true podemos afirmar que ese estado es alcanzable y que
          por tanto podemos llegar a tener C litros en una u otra vasija. De nuevo por la 
          observación dos, no vale mirar únicamente la última tabla.
    \item Sobre la reconstrucción de la solución: \\
          Esta es tal vez la parte más complicada. ¿Cómo reconstruir la solución con la 
          representación adoptada? Muy fácil, con otra representación ;). Si le añadimos
          a cada casilla marcada como true en la matriz del estado k un código de la operación
          que nos trajo hasta este punto o, directamente el estado anterior a la acción ejecutada
          el algoritmo cae por su propio peso a cambio de un pequeño espacio adicional en memoria.
    \item Sobre el coste: \\
          En el caso peor tendremos que llenar todas las tablas porque en ninguna iteración se
          da la condición de parada. En este caso, el algoritmo tiene un orden de complejidad
          \textit{O($k \times C_1 \times C_2$)} pues aunque en los casos de mayor complicación
          se pueden ejecutar sobre cada estado hasta 6 acciones, consideramos la ejecución de 
          estas de coste constante.
    
\end{itemize}

%\begin{listing}
%//Aplicable a la clase Grafo.h (ver implementaciones dadas) 
%\end{listing}


\end{document}