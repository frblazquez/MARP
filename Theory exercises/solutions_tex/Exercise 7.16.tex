\documentclass[12pt]{article}
\usepackage[margin=1in]{geometry} 
\usepackage{amsmath,amsthm,amssymb,amsfonts}

\newcommand{\N}{\mathbb{N}}
\newcommand{\Z}{\mathbb{Z}}
 
\newenvironment{problem}[2][Problema]{\begin{trivlist}
\item[\hskip \labelsep {\bfseries #1}\hskip \labelsep {\bfseries #2.}]}{\end{trivlist}}

\usepackage[utf8]{inputenc}                 % Using Unicode encoding
\usepackage[spanish]{babel}                 % Writing in Spanish language
%\usepackage{listings}                       % For including C++ code
\setlength{\parindent}{0mm}                 % Paragraph indentation null
\setlength{\parskip}{\baselineskip}         % Distance among paragraphs

%--------------------------------------------------------------------------
%   PARA INCLUIR CÓDIGO C++ EN EL DOCUMENTO
%--------------------------------------------------------------------------
\usepackage{color}
\definecolor{gray97}{gray}{.97}
\definecolor{gray75}{gray}{.75}
\definecolor{gray45}{gray}{.45}

\usepackage{listings}
\lstset{ frame=Ltb,
framerule=0pt,
aboveskip=0.5cm,
framextopmargin=3pt,
framexbottommargin=3pt,
framexleftmargin=0.4cm,
framesep=0pt,
rulesep=.4pt,
backgroundcolor=\color{gray97},
rulesepcolor=\color{black},
%
stringstyle=\ttfamily,
showstringspaces = false,
basicstyle=\small\ttfamily,
commentstyle=\color{gray45},
keywordstyle=\bfseries,
%
numbers=left,
numbersep=15pt,
numberstyle=\tiny,
numberfirstline = false,
breaklines=true,
}

% minimizar fragmentado de listados
\lstnewenvironment{listing}[1][]
{\lstset{#1}\pagebreak[0]}{\pagebreak[0]}

\lstdefinestyle{consola}
{basicstyle=\scriptsize\bf\ttfamily,
backgroundcolor=\color{gray75},
}

\lstdefinestyle{C}
{language=C,
}

\begin{document}

%---------------------------------------------------------------------------------------
% TÍTULO Y AUTOR
%---------------------------------------------------------------------------------------
\title{\textbf{Hoja 7 - Ejercicio 16}}
\author{Francisco Javier Blázquez Martínez}
\date{}
\maketitle
 
%---------------------------------------------------------------------------------------
% ENUNCIADO DEL PROBLEMA
%---------------------------------------------------------------------------------------
\begin{problem}{7.16}
Juanita afronta sus exámenes finales del curso con precipitación. Para cada asignatura
$A_i$ del curso conoce los días $f_i$ que quedan hasta la fecha de su examen, y sabe la 
cantidad de días $d_i$ que debería dedicar a su estudio (lógicamente, antes del 
correspondiente examen) para poder aprobarla. Juanita se teme que no podrá con todas las 
asignaturas, así que ha dado un valor $v_i$ a cada una de ellas, y ha de determinar qué 
asignaturas estudiará, y en qué orden, hasta la finalización de los exámenes, a sabiendas 
de que si dedica a una asignatura menos de los $d_i$ estipulados la suspenderá 
irremediablemente. Debéis ayudar a Juanita a elegir las asignaturas que estudiará, de 
manera que maximice el valor total de las asignaturas que ha decidido estudiar con tiempo 
suficiente para aprobarlas. Describir el algoritmo de ramificación y poda apropiado para
este problema, e indicar con detalle las cotas inferior y superior empleadas.
\end{problem}

%---------------------------------------------------------------------------------------
% SOLUCIÓN
%---------------------------------------------------------------------------------------
\textbf{Solución:} \\

%\setlength{\parskip}{0mm}
%\setlength{\parskip}{\baselineskip}

\setlength{\parskip}{0mm}
Observaciones preliminares:

\begin{itemize}
    \item En las transparencias del tema de ramificación y poda, aparece una serie de 
          diapositivas con el título ``\textit{tareas con plado fijo, duración y coste}''.
          La idea es muy similar.
    \item Para simplificar los cálculos de la factibilidad o infactibilidad de las soluciones
          parciales usamos la idea dada en las transparencias. Suponemos los exámenes 
          ordenados cronológicamente.
\end{itemize}
\setlength{\parskip}{\baselineskip}


Usamos la siguente notación: \\ \setlength{\parskip}{0mm}

$f_1, f_2, f_3, ... , f_n$ \hspace{1cm} Fechas de los exámenes. \\
$d_1, d_2, d_3, ... , d_n$ \hspace{1cm} Días necesarios de preparación de cada examen. \\
$p_1, p_2, p_3, ... , p_n$ \hspace{1cm} Puntuación asociada a cada examen. \\
    
maxPuntos(\{k, ... , n\}, $f_{end}$) \\ 

Esta última función es para poder calcular los máximos puntos que puedo conseguir con los 
exámenes del k al n si estoy disponible para estudiar a partir de la fecha $f_{end}$. 
Entonces:

\setlength{\parskip}{\baselineskip}

maxPuntos(\{k, ... , n\}, $f_{end}$) = máximo entre \{ \\
\setlength{\parindent}{10mm} 
\indent maxPuntos(\{k+1, ... , n\}, $f_{end}$), \\
\indent maxPuntos(\{k+1, ... , n\}, $f_{end}$ + $d_k$) + $p_k$ \\ \} \\
\setlength{\parindent}{0mm} 

Tenemos ya entonces la recurrencia para el esquema de ramificación y poda. Vamos a aplicar el
esquema optimista/pesimista visto en clase junto con una cola de prioridad de máximos para 
simplificar la condición de parada y explorar siempre por los nodos más prometedores.

Optamos para tener una cota optimista sencilla suponer que todos los exámenes restantes pueden
prepararse. Para esto calculamos y almacenamos en un vector la suma de puntuaciones de los 
exámenes a partir de un cierto nivel (ver ``calculaOptimistas''). Para la cota pesimista optamos
por completar de forma voraz nuestra solución. 

\begin{listing}
// Para la cota optimista
proc calculaOptimistas(p[])
{
    optimistas[n] = p[n];
    
    for(i = n-1; i > 0; i--)
        optimistas[i] = optimistas[i+1] + p[i];
}


// Para la cota pesimista
dev:pes pesimista(f[], d[], p[], k, n, f_end)
{
    pes = 0;
    
    for(int i = k; i <=n; i++)
    {
        if(f[i] >= f_end + d[i])
        {
            pes   += p[i];
            f_end += d[i];
        }
    }
}
\end{listing}

A partir de aquí es el esquema clásico visto en clase. Tener una variable con la mejor solución
hasta el momento (que se actualizará con la cota pesimista) y una cola de prioridad con los
nodos con soluciones parciales explorados hasta el momento. La recursión del inicio de esta hoja
nos da la forma de prolongar estas soluciones parciales y, por último, la condición de parada 
será que la cota opt del siguente nodo sea menor o igual que nuestra mejor solución. 
\end{document}