\documentclass[12pt]{article}
\usepackage[margin=1in]{geometry} 
\usepackage{amsmath,amsthm,amssymb,amsfonts}

\newcommand{\N}{\mathbb{N}}
\newcommand{\Z}{\mathbb{Z}}
 
\newenvironment{problem}[2][Problema]{\begin{trivlist}
\item[\hskip \labelsep {\bfseries #1}\hskip \labelsep {\bfseries #2.}]}{\end{trivlist}}

\usepackage[utf8]{inputenc}                 % Using Unicode encoding
\usepackage[spanish]{babel}                 % Writing in Spanish language
%\usepackage{listings}                       % For including C++ code
\setlength{\parindent}{0mm}                 % Paragraph indentation null
\setlength{\parskip}{\baselineskip}         % Distance among paragraphs

%--------------------------------------------------------------------------
%   PARA INCLUIR CÓDIGO C++ EN EL DOCUMENTO
%--------------------------------------------------------------------------
\usepackage{color}
\definecolor{gray97}{gray}{.97}
\definecolor{gray75}{gray}{.75}
\definecolor{gray45}{gray}{.45}

\usepackage{listings}
\lstset{ frame=Ltb,
framerule=0pt,
aboveskip=0.5cm,
framextopmargin=3pt,
framexbottommargin=3pt,
framexleftmargin=0.4cm,
framesep=0pt,
rulesep=.4pt,
backgroundcolor=\color{gray97},
rulesepcolor=\color{black},
%
stringstyle=\ttfamily,
showstringspaces = false,
basicstyle=\small\ttfamily,
commentstyle=\color{gray45},
keywordstyle=\bfseries,
%
numbers=left,
numbersep=15pt,
numberstyle=\tiny,
numberfirstline = false,
breaklines=true,
}

% minimizar fragmentado de listados
\lstnewenvironment{listing}[1][]
{\lstset{#1}\pagebreak[0]}{\pagebreak[0]}

\lstdefinestyle{consola}
{basicstyle=\scriptsize\bf\ttfamily,
backgroundcolor=\color{gray75},
}

\lstdefinestyle{C}
{language=C,
}

\begin{document}
 
\title{\textbf{Hoja 4 - Ejercicio 13}}
\author{Francisco Javier Blázquez Martínez}
\date{}
\maketitle
 
\begin{problem}{4.13.1}
Dado un grafo G representado de forma abstracta por medio de una funciónn que a cada
vértice del grafo le asocia una lista de vértices adyacentes, escribir un algoritmo
que calcule el número de componentes conexas de G y cuántas de ellas no tienen ciclos.
\end{problem}

\setlength{\parskip}{0mm}
Observaciones preliminares:

\begin{itemize}
    \item Ya conocemos algoritmos para calcular el número de componentes conexas de 
          un grafo (basados en DFS o BFS, ver apuntes de la asignatura).
    \item Al ser un grafo no dirigido, no vamos a considerar ciclos de una única 
          arista. De lo contrario toda componente conexa tiene ciclos si y sólo si 
          tiene más de un vértice.
\end{itemize}

La duda entonces es si podemos modificar ligeramente los algorimtos de recorrido BFS
o DFS para conseguir lo que nos piden, calcular el número de componentes conexas y al
mismo tiempo saber cuáles de estas contienten ciclos. La respuesta es que en efecto
es posible. En nuestro recorrido, guardaremos para cada vértice nuevo de nuestra 
componente conexa el vértice que nos permitió llegar hasta él. Si en algún momento del
recorrido podemos llegar a él por otro vértice distinto, el hecho de estar los tres
en la misma componente conexa nos permite asegurar que forman un ciclo. El código 
sería entonces:

\setlength{\parskip}{\baselineskip}

\begin{listing}
//Aplicable a la clase Grafo.h (ver implementaciones dadas) 

 void componentesConexasCiclos(int &conexas, int &ciclos)
 {
    conexas = 0;
    ciclos = 0;
    bool tieneCiclos;
    
    int padres[_V]; 
    for(int i = 0; i < _V; i++) padres[i] = -1;
    
    for(int i = 0; i < _V; i++)
    {
        if(padres[i] == -1)
        {
        std::vector<int> siguientes;
        conexas++;
        tieneCiclos = false;
        padres[i] = i;
        siguientes.push_back(i);
        
        while(!siguientes.empty())
        {
            int v = siguientes.back();
            int adyacente;
            siguientes.pop_back();
                    
            for(int j = 0; j < _ady[v].size(); j++)
            {
                adyacente = _ady[v][j];
                
                if(padres[adyacente] == -1)
                {
                    padres[adyacente] = v;
                    siguientes.push_back(adyacente);
                }
                else if(padres[v] != adyacente)
                {
                    if(!tieneCiclos)
                    {
                        tieneCiclos = true;
                        ciclos++;
                    }
                }
            }
        } // end while
        } // end if
    }       
 }
\end{listing}

Vemos aquí que el array de enteros \textit{padres} nos sirve tanto para marcar los
vértices que ya hemos visitado como para marcar el vértice que nos permitió alcanzar
este vértice en la componente conexa y prevenir ciclos. Esto es, padres[i] = v implica
que llegamos al vértice i gracias a que está conectado con el vértice v. El vértice
de inicio del recorrido en una componente conexa se marca como su propio padre para
que el algoritmo sepa que ya se ha incluido en la componente conexa a explorar, esto
no quiere decir que permitamos autoaristas en el grafo, es un detalle de 
implementación fijado de forma arbitraria. El hecho de inicializar el array de padres
al valor -1 es también arbitrario, siendo válido cualquier valor que no se encuentre
en el rango [0, \_V).


\begin{problem}{4.13.2}
Indicar como encontrar de manera eficiente un subconjunto de arcos mínimo que mantenga
la relación de alcanzabilidad original.
\end{problem}

Si nos fijamos, al guardar para cada vértice que añadimos a la componente conexa el
vértice desde el que llegamos a este, estamos de hecho guardando la arista que une
el vértice que acabamos de añadir a la componente conexa con el vértice que nos 
permite llegar a este. El array \textit{padres} no nos vale únicamente para prevenir
ciclos sino que nos da al mismo tiempo un árbol de recubrimiento mínimo. Nos vale
por tanto la solución del apartado anterior y basta con interpretar el array
\textit{padres}.El subconjunto de arcos pedido es por tanto:

\setlength{\parskip}{0mm}


    \[S = \{(i,padres[i]) / padres[i] != i\}\]

\end{document}