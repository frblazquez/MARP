\documentclass[12pt]{article}
\usepackage[margin=1in]{geometry} 
\usepackage{amsmath,amsthm,amssymb,amsfonts}

\newcommand{\N}{\mathbb{N}}
\newcommand{\Z}{\mathbb{Z}}
 
\newenvironment{problem}[2][Problema]{\begin{trivlist}
\item[\hskip \labelsep {\bfseries #1}\hskip \labelsep {\bfseries #2.}]}{\end{trivlist}}

\usepackage[utf8]{inputenc}                 % Using Unicode encoding
\usepackage[spanish]{babel}                 % Writing in Spanish language
%\usepackage{listings}                       % For including C++ code
\setlength{\parindent}{0mm}                 % Paragraph indentation null
\setlength{\parskip}{\baselineskip}         % Distance among paragraphs

%--------------------------------------------------------------------------
%   PARA INCLUIR CÓDIGO C++ EN EL DOCUMENTO
%--------------------------------------------------------------------------
\usepackage{color}
\definecolor{gray97}{gray}{.97}
\definecolor{gray75}{gray}{.75}
\definecolor{gray45}{gray}{.45}

\usepackage{listings}
\lstset{ frame=Ltb,
framerule=0pt,
aboveskip=0.5cm,
framextopmargin=3pt,
framexbottommargin=3pt,
framexleftmargin=0.4cm,
framesep=0pt,
rulesep=.4pt,
backgroundcolor=\color{gray97},
rulesepcolor=\color{black},
%
stringstyle=\ttfamily,
showstringspaces = false,
basicstyle=\small\ttfamily,
commentstyle=\color{gray45},
keywordstyle=\bfseries,
%
numbers=left,
numbersep=15pt,
numberstyle=\tiny,
numberfirstline = false,
breaklines=true,
}

% minimizar fragmentado de listados
\lstnewenvironment{listing}[1][]
{\lstset{#1}\pagebreak[0]}{\pagebreak[0]}

\lstdefinestyle{consola}
{basicstyle=\scriptsize\bf\ttfamily,
backgroundcolor=\color{gray75},
}

\lstdefinestyle{C}
{language=C,
}

\begin{document}
 
\title{\textbf{Hoja 3 - Ejercicio 4}}
\author{Francisco Javier Blázquez Martínez}
\date{}
\maketitle
 
\begin{problem}{3.4}
Enriquecer un montículo de Williams con las operaciones \textit{decrecerClave} y 
\textit{aumentarClave} que disminuyan o aumenten el valor de una clave dada su
posición en el vector.
\end{problem}

Vamos a considerar el caso en el que estemos trabajando sobre un montículo de mínimos,
siendo el caso de un montículo de máximos totalmente análogo. Se nos pide entonces
incluir las operaciones de decrecer y aumentar clave (dada una posición en el vector)
y que estas sean de coste logarítmico. Nosotros vamos a englobar en nuestra solución
ambas operaciones en una única que sea \textit{cambiarClave}, que reciba una posición
y una nueva clave y modifique este elemento del montículo cambiando su clave al nuevo
valor recibido como parámetro. Es claro que la solución inocente de únicamente cambiar
el valor no es válida pues puede suceder (y es de hecho el caso más común) que al 
cambiar la clave estemos rompiendo la estructura de montículo de mínimos. Esto es, la
equivalencia entre vector - montículo de mínimos puede romperse. 

La solución pasa entonces por \textit{flotar} o \textit{hundir} el elemento al que
cambiamos la clave según corresponda (tomando las operaciones flotar e hundir vistas
en clase, cuyo coste sabemos logarítmico). El código correspondiente sería:

\begin{listing}
 template<typename T>
 void cambiarClave(int position, T newKey)
 {
    array[position] = newKey;
    
    flotar(position);
    hundir(position);
 }
\end{listing}
 
 \setlength{\parskip}{0mm} 
 Observaciones:
 
\begin{enumerate}
    \item Se ha partido de la clase \textit{PriorityQueue} a disposición como código
          de la asignatura.
    \item Se llama tanto a flotar como a hundir independiente de si se aumenta o 
          disminuye la clave sabiendo que, una de las llamadas terminará de forma
          inmediata sin modificar el vector.
\end{enumerate}
 


\end{document}