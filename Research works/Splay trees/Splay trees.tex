%   Copyright © 2019
%
%   Francisco Javier Blázquez Martínez ~ frblazqu@ucm.es
%
%   Double degree in Mathematics-Computer engineering.
%   Complutense university, Madrid.

%--------------------------------------------------------------------------
% CABECERA DEL DOCUMENTO - IMPORTACIÓN DE PAQUETES
%--------------------------------------------------------------------------
\documentclass[letterpaper,12pt]{article}
\usepackage{tabularx} % extra features for tabular environment
\usepackage{amsmath}  % improve math presentation
\usepackage{graphicx} % takes care of graphic including machinery
\usepackage[margin=1in,letterpaper]{geometry} % decreases margins
\usepackage{cite} % takes care of citations
\usepackage[final]{hyperref} % adds hyper links inside the generated pdf file
\hypersetup{
	colorlinks=true,       % false: boxed links; true: colored links
	linkcolor=blue,        % color of internal links
	citecolor=blue,        % color of links to bibliography
	filecolor=magenta,     % color of file links
	urlcolor=blue         
}

\usepackage[utf8]{inputenc}                 % Using Unicode encoding
\usepackage[spanish]{babel}                 % Writing in Spanish language
%\usepackage{listings}                       % For including C++ code
\setlength{\parindent}{0mm}                 % Paragraph indentation null
\setlength{\parskip}{\baselineskip}         % Distance among paragraphs

%--------------------------------------------------------------------------
%   PARA INCLUIR CÓDIGO C++ EN EL DOCUMENTO
%--------------------------------------------------------------------------
\usepackage{color}
\definecolor{gray97}{gray}{.97}
\definecolor{gray75}{gray}{.75}
\definecolor{gray45}{gray}{.45}

\usepackage{listings}
\lstset{ frame=Ltb,
framerule=0pt,
aboveskip=0.5cm,
framextopmargin=3pt,
framexbottommargin=3pt,
framexleftmargin=0.4cm,
framesep=0pt,
rulesep=.4pt,
backgroundcolor=\color{gray97},
rulesepcolor=\color{black},
%
stringstyle=\ttfamily,
showstringspaces = false,
basicstyle=\small\ttfamily,
commentstyle=\color{gray45},
keywordstyle=\bfseries,
%
numbers=left,
numbersep=15pt,
numberstyle=\tiny,
numberfirstline = false,
breaklines=true,
}

% minimizar fragmentado de listados
\lstnewenvironment{listing}[1][]
{\lstset{#1}\pagebreak[0]}{\pagebreak[0]}

\lstdefinestyle{consola}
{basicstyle=\scriptsize\bf\ttfamily,
backgroundcolor=\color{gray75},
}

\lstdefinestyle{C}
{language=C,
}

%--------------------------------------------------------------------------
% INICIO DEL DOCUMENTO - TÍTULO, AUTOR, FECHA ...
%--------------------------------------------------------------------------
\begin{document}

\title{\bf{Implementación detallada de Splay Trees}}
\author{Francisco Javier Blázquez Martínez}
\date{}
\maketitle

%--------------------------------------------------------------------------
% RESUMEN INICIAL
%--------------------------------------------------------------------------
\begin{abstract}

Trabajo realizado en el marco de la asignatura \textit{Métodos algorítmicos 
en resolución de problemas}. El objetivo de este documento es el de explicar
las decisiones de diseño tomadas en el desarrollo del código y aclarar 
distintos aspectos de las implementaciones en \textit{C++} y \textit{Java} 
que puedan generar confusión.

\end{abstract}

%--------------------------------------------------------------------------
%INICIO DEL CUERPO DEL DOCUMENTO
%--------------------------------------------------------------------------
\section{Introducción:}

Los árboles binarios de búsqueda autoajustables (\textit{splay trees}) 
aparecen en 1985 ideados por Robert Tarjan y Daniel Selator.
Estos no son sino árboles binarios de búsqueda con una peculiaridad, una 
operación interna sobre los nodos llamada ``\textit{splay}''. Esta operación, 
invisible en todo momento al usuario, recibe un nodo como parámetro y 
transforma el árbol de partida en un árbol binario de búsqueda equivalente 
(manteniendo exactamente los elementos iniciales) en el cual la raíz es el 
nodo seleccionado. Intuitivamente, esta operación se emplea para conseguir 
que elementos recientemente requeridos se encuentren cerca de la raíz y, por 
tanto, sea más rápido volver a acceder a ellos. De aquí proviene su apellido 
``\textit{autoajustable}'' pues esta operación de splay consigue que el árbol 
de preferencia en alturas menores a elementos más empleados en cualquier 
secuencia de operaciones.

El principal inconveniente de esta estructura de datos, sin embargo, es que, 
a diferencia de los árboles AVL o los árboles Red-Black, no se mantiene 
ningún invariante en la altura del árbol, pudiendo llegar a ser esta del 
orden del número de elementos. Como ya sabemos, búsqueda, insercción y 
borrado en un árbol binario son operaciones del orden de la altura del árbol.
Esto quiere decir que el coste en el caso peor de estas operaciones en splay 
trees es \textit{O(n)}. A pesar de esto, y precisamente el hecho de su 
publicación por R. Tarjan y D. Selator, se puede demostrar que en cualquier 
secuencia de inserciones, borrados y búsquedas de elementos estas operaciones
tienen un coste amortizado logarítmico. De hecho, se puede aprovechar esta 
propiedad de los splay trees de que elementos accedidos recientemente son 
fácilmente accesibles para conseguir una mayor eficiencia bajo ciertas 
circunstancias (no aleatorias). Aquí nos limitaremos a explicar los detalles 
de la implementación presentada, para una mayor comprensión recomiendo 
consultar la bibliografía al final del documento.

%\cite{melissinos, Cyr, Wiki}   % Para tener un ejemplo de cómo citar

\section{Implementación:}

Como ya hemos mencionado, los splay trees son esencialmente árboles de
búsqueda binarios y, como tales, tienen infinidad de aplicaciones. Es preciso
antes de comenzar la implementación concretar las operaciones que vamos a 
permitir ejecutar al usuario. En mi caso he optado por aplicar los splay 
trees para crear una estrucura de datos de conjunto ordenado de elementos de
tipo genérico, con un orden que puede ser especificado por el usuario.
La interfaz pública de usuario se puede consultar en la tabla inferior.

\begin{center}
\begin{tabular}{|l|c|c|l|}

	\hline
	\textbf{Operación} & \textbf{Coste} & \textbf{Caso peor} & \textbf{Descripción}\\
	\hline
    splay\_tree()    & \textit{O(1)}     & \textit{O(1)} & Crea un conjunto vacío\\
	\hline
    \~splay\_tree()  & \textit{O(n)}     & \textit{O(n)} & Vacía el conjunto y libera la memoria\\ 
	\hline
	bool empty()     & \textit{O(1)}     & \textit{O(1)} & Devuelve true si el conjunto está vacío\\
	\hline
	bool insert(elem)& \textit{O(log(n))}& \textit{O(n)} & Devuelve true si inserta el elemento\\
	\hline
	bool erase(elem) & \textit{O(log(n))}& \textit{O(n)} & Devuelve true si elimina el elemento\\
	\hline
	bool find(elem)  & \textit{O(log(n))}& \textit{O(n)} & Devuelve true si \textit{elem} está en el conjunto\\
	\hline
	void print()     & \textit{O(n)}     & \textit{O(n)} & Muestra en preorder los elementos y su altura\\
	\hline
	
\end{tabular}
\end{center}

El diseño se ha realizado así para asemejarse a la clase \textit{set} de la 
librería estándar de \textit{C++}. De hecho, que detrás de esta interfaz se 
encuentra la estructura de datos de splay tree es algo invisible al usuario. 
Una diferencia a destacar con respecto la clase set de la \textit{C++ STL} es 
que se incluye el método \textit{print()}. Esto es para poder visualizar en 
consola la estructura completa del conjunto, cómo se almacenan y cómo varía la
posición de los nodos al ejecutar las distintas operaciones, más allá del uso 
como conjunto de la clase. Cabe destacar también de la implementación las 
siguientes características:

\begin{enumerate}
\setlength{\parskip}{0mm}         % Distance among paragraphs

\item No se permite la inserción repetida de un mismo elemento, esto es, 
      nunca nuestro splay tree subyacente al conjunto implementado contendrá
      dos nodos con elementos iguales. Para garantizar esto es necesario 
      también que se de el punto número dos.
\item El usuario puede decidir el orden que rige la estructura siempre que 
      este constituya un orden total sobre el conjunto de posibles 
      elementos. La estructura implementada está preparada para funcionar con 
      un comparador cualquiera pero puede llegar a estados inconsistentes si 
      esta condición no se satisface.
\item A diferencia de otras implementaciones de splay trees, no se contempla
      la unión de árboles/conjuntos en la estructura implementada. Esto tiene
      importancia en los métodos de inserción y borrado de elementos.
\item La estructura de splay tree que subyace a la implementación de conjunto
      puede ser implementada sin añadir atributos adicionales a los nodos del
      árbol (más allá del elemento y los punteros a ambos hijos). Por una 
      mayor sencillez en la implementación he optado por añadir un atributo,
      un puntero al nodo padre para poder actualizar sus atributos de forma 
      sencilla en \textit{O(1)}.
     
\end{enumerate}

\section{Algoritmos:}
\subsection{Atributos de la clase:}

Como ya hemos dicho, hemos optado por añadir como atributo propio de cada nodo
un puntero a su nodo padre para simplificar el código. También hemos optado
por que tanto la clase nodo (invisible al usuario) como los atributos de la 
clase sean \textit{protected} y no \textit{private} en vista a posibles 
modificaciones futuras y extensiones de funcionalidad mediante herencia. Abajo
se muestra la declaración de la clase sin métodos.

\begin{lstlisting}
template<typename T, typename Comp = std::less<T>>
class splay_tree
{
  protected:
    struct Node
    {
      T     elem;
      Node* left;
      Node* right;
      Node* parent;

      Node(Node* l, T const& e, Node* r, Node* p) 
          :elem(e),left(l),right(r),parent(p) {}
    };

    Node* root;
    Comp  lower;
}; //splay_tree
\end{lstlisting}
\subsection{Búsqueda:}

Se realiza igual que en el caso de un árbol binario de búsqueda convencional 
con la salvedad de que se realiza un splay del nodo encontrado o el último 
nodo visitado si el elemento no se encuentra en el árbol. Recuerdo que 
\textit{lower} es un comparador que actua sobre los elementos de tipo 
\textit{T} (ver el código añadido abajo). Se aprecia en este método por qué el 
comparador debe crear un orden total en el conjunto de posibles valores de 
\textit{T}. En el bucle de descenso desde la raíz, buscando el elemento 
recibido como parámetro, se realiza una comparación nodo a nodo. Si el 
parámetro es menor que el elemento del nodo, la búsqueda continúa por el hijo 
izquierdo del nodo. Si el parámetro es mayor, continúa por el hijo derecho. Si 
un elemento no es mayor ni menor que el recibido como parámetro, el método 
sobreentiende que es igual. Es por esto que el comparador debe establecer un 
orden total estricto. 

\begin{lstlisting}
template<typename T, typename Comp>
bool splay_tree<T, Comp>::find(T const& elem)
{
  Node* father = nullptr; Node* node = root;

  while(node != nullptr)
  {
    father = node;

         if(lower(elem, node->elem))     node = node->left;
    else if(lower(node->elem, elem))     node = node->right;
    else {splay(node); return true;} // Encuentra el elemento
  }

  if(father != nullptr) splay(father);

  return false;
}
\end{lstlisting}
\subsection{Inserción:}

La inserción es igual que en un árbol binario de búsqueda con la salvedad de 
que al terminar el proceso se realiza una llamada al método splay con el nodo 
insertado, de realizarse la inserción, o con aquel nodo que ya contiene la 
clave a insertar. No incluir la función de unión de árboles descarta la 
posibilidad de reducir la inserción de elementos a la unión del árbol actual
con un árbol con un único nodo, el del elemento a insertar. Sí que se podía
optar por una implementación recursiva en vez de iterativa a la hora de 
descender por los nodos del árbol. En mi caso he optado por una implementación
iterativa pues en la implementación recursiva, al llegar a un puntero nulo al
final del descenso no se puede saber cuál es su nodo padre si no es añadiendo
un atributo más como parámetro. La implementación recursiva alternativa 
desarrollada puede verse en el fichero \textit{splay\_tree.h} como comentario
en el código.

\begin{lstlisting}
template<typename T, typename Comp>
bool splay_tree<T, Comp>::insert(T const& elem)
{
  Node* father = nullptr; Node* node = root;

  while(node != nullptr)
  {
    father = node;

         if(lower(elem, node->elem))     node = node->left;
    else if(lower(node->elem, elem))     node = node->right;
    else {splay(node); return false;} // El elemento esta
  }

  // Llegados aqui el elemento no esta, lo insertamos
  node = new Node(nullptr, elem, nullptr, father);

  // Actualizamos su padre y, si es necesario la raiz
  if(father != nullptr)
    {if(lower(elem, father->elem)) father->left = node;
     else                          father->right= node;}
  else
    root = node;

  // Finalmente llevamos el nuevo nodo a la raiz
  splay(node);  return true;
}
\end{lstlisting}
\subsection{Borrado:}

Al igual que en los métodos anteriores, la filosofía es la misma que en los
árboles binarios convencionales. De nuevo la diferencia la marca la llamada al
método splay al terminar el borrado. En algunas variantes de splay trees, se 
opta por flotar primero el nodo a borrar a la raíz y luego borrarlo. En mi 
caso he optado por primero borrar el elemento (de estar en el árbol) y
entonces flotar su nodo padre. El elemento borrado se reemplaza por el menor 
elemento del conjunto mayor que el del nodo a borrar para mantener la 
estructura de árbol binario de búsqueda. Si el elemento a borrar no se 
encuentra en el conjunto se llama al método splay con el último nodo visitado 
durante la búsqueda del elemento.

\begin{lstlisting}
template<typename T, typename Comp>
bool splay_tree<T, Comp>::erase(T const& elem)
{
  Node* father = nullptr; Node* node = root;

  while(node)
  {
        if(lower(elem, node->elem)) 
            {father = node; node = node->left;}
   else if(lower(node->elem, elem)) 
            {father = node; node = node->right;}
   else break; // Llegamos al nodo a borrar
  }

  /*   ACLARACIONES:
  *
  *    A partir de este momento node apunta al nodo que queremos
  *    borrar, vamos a distinguir casos segun su situacion.
  *
  * -> Los punteros se pueden interpretar como booleanos. En C++ 
  *    un puntero equivale a falso si y solo si es nulo.
  * -> El unico nodo con padre nulo es la raiz. La condicion
  *    (father == nullptr) equivale a (node == root).
  */
  
    // Si el elemento no esta, flotamos el padre y terminamos
  if(node == nullptr) {if(father) splay(father); return false;}
  
  if(!node->left && !node->right)   // CASO 1: Hoja
  {
    if(node->parent == nullptr) root = nullptr;
    else
    {
      if(node == node->parent->left)  node->parent->left = nullptr;
      else                            node->parent->right= nullptr;
    }
  }
  else if(!node->left)              // CASO 2: Solo hijo derecho
  {
    if(node->parent == nullptr) 
        {root = node->right; root->parent = nullptr;}
    else
    {
      node->right->parent = node->parent;
      if(node == node->parent->left)  
        node->parent->left = node->right;
      else                            
        node->parent->right= node->right;
    }
  }
  else if(!node->right)             // CASO 3: Solo hijo izquierdo
  {
      if(node->parent == nullptr) 
          {root = node->left;  root->parent = nullptr;}
      else
      {
        node->left->parent = node->parent;
        if(node == node->parent->left)  
            node->parent->left = node->left;
        else                            
            node->parent->right= node->left;
      }
  }
  else                              // CASO 4: Ambos hijos existen
  {
    // Hallamos el elemento que reemplace el nodo a borrar
    Node* lessGreater = node->right; 
    while(lessGreater->left) lessGreater = lessGreater->left;

    // Lo desconectamos de su posicion
    if(lessGreater->right)
    {
      lessGreater->right->parent = lessGreater->parent;

      if(lessGreater == lessGreater->parent->right) 
            lessGreater->parent->right = lessGreater->right;
      else                                          
            lessGreater->parent->left  = lessGreater->right;
    }
    else
    {
      if(lessGreater == lessGreater->parent->right)  
            lessGreater->parent->right = nullptr;
      else  
            lessGreater->parent->left  = nullptr;
    }

    // Lo conectamos en la posicion del nodo a borrar
    lessGreater->left  = node->left;   
    lessGreater->right = node->right;  
    lessGreater->parent= node->parent;
    
    if(node->left)   node->left->parent = lessGreater;
    if(node->right)  node->right->parent= lessGreater;
    
    if(node->parent)
      {if(node == node->parent->left) 
            node->parent->left = lessGreater;
       else                           
            node->parent->right= lessGreater;}
    else
      root = lessGreater;
  }

  delete node;  if(father) splay(father);
  return true;
}
\end{lstlisting}
\subsection{Rotación de un nodo:}

Se contemplan rotaciones hacia la izquierda o hacia la derecha en función
de si el nodo es reemplazado por su hijo derecho o izquierdo. Estas son 
iguales a las rotaciones de los árboles AVL u otros árboles binarios 
balanceados pero, claro está, actualizando también el atributo añadido a los
nodos, el nodo padre al que apuntan. Se incluye el código correspondiente a 
una rotación hacia la izquierda, siendo el código de la rotación izquierda
simétrico.

\begin{lstlisting}
template<typename T, typename Comp>
void splay_tree<T, Comp>::rotateRight(Node* node)
{
  if(node == nullptr)
    throw std::domain_error("Right rotation over null node!");

  Node* father    = node->parent;   //Guardamos el padre
  Node* leftChild = node->left;     //Guardamos el hijo izquierdo

  if(leftChild != nullptr)
  {
    node->left        = leftChild->right;
    node->parent      = leftChild;
    leftChild->parent = father;
    leftChild->right  = node;

    if(node->left != nullptr)  node->left->parent = node;
    if(father     != nullptr)
      {if(node == father->left)  father->left = leftChild;
       else                      father->right= leftChild;}
    else
      root = leftChild;
  }
}
\end{lstlisting}
\subsection{Splay:}

El algoritmo lleva el nodo recibido como parámetro a la raiz manteniendo la
estructura de árbol binario de búsqueda. Para esto, mientras el nodo no se sitúe en la raiz, si su nodo padre es la raíz, realiza una única rotación 
para ocupar la posición de su nodo padre (rotación izquierda o derecha según
corresponda). Si por contra el nodo tiene un ``nodo abuelo'' combina dos 
rotaciones que llevan el el nodo a flotar a dos alturas menos. Al no hacer uso 
del comparador, tanto este método como las rotaciones que emplea son 
consistentes a pesar de que el orden inferido del comparador no sea total.


\begin{lstlisting}
template<typename T, typename Comp>
void splay_tree<T, Comp>::splay(Node* node)
{
  if(node == nullptr)
    throw std::domain_error("Splay over null node!");

  while(node->parent)
  {
    if(!node->parent->parent)  // Hijo directo de la raiz
    {
      if(node->parent->left == node)  rotateRight(node->parent);
      else                            rotateLeft(node->parent);
    }
    else                       // El nodo tiene nodo abuelo
    {
      Node* grandParent = node->parent->parent;

      if(grandParent->left       && node==grandParent->left->left)
       {rotateRight(grandParent);
        rotateRight(node->parent);}
      else if(grandParent->right && node==grandParent->right->right)
        {rotateLeft(grandParent);
         rotateLeft(node->parent);}
      else if(grandParent->left  && node==grandParent->left->right)
        {rotateLeft(node->parent);
         rotateRight(grandParent);}
      else if(grandParent->right && node==grandParent->right->left)
        {rotateRight(node->parent);
         rotateLeft(grandParent);}
      else
        throw "Illegal state reached during splay operation!";
    }
  }
}
\end{lstlisting}

\section{Tests:}
Por mayor sencillez a la hora de comprobar el correcto funcionamiento de la 
clase he incluido dos tests, uno manual y otro automático. Esta sección tiene
el objetivo de explicar el funcionamiento de estos para que puedan ser 
ejecutados por cualquier persona con interés en apreciar los múltiples cambios
de estructura de los splay trees en una secuencia de operaciones. Se puede
también fácilmente incluir mediciones del tiempo de ejecución en el test 
automático para convencerse de que el coste amortizado de las operaciones es 
logarítmico.

Se incluye abajo el código correspondiente para ejecutar un test manual de la
clase \textit{splay\_tree}. Este test lee por consola un código de operación
y ejecuta la operación correspondiente:

\setlength{\parskip}{0mm}         % Distance among paragraphs
\begin{itemize}

\item -1 para terminar el test.
\item  0 para mostrar el árbol.
\item  1 para buscar un elemento (se lee a continuación).
\item  2 para insertar un elemento (se lee a continuación).
\item  3 para borrar un elemento (se lee a continuación).
     
\end{itemize}
\setlength{\parskip}{\baselineskip}         % Distance among paragraphs

\begin{lstlisting}
#include "splay_tree.h"
#include <cstdlib>
#include <time.h>
using namespace std;

int main()
{
  splay_tree<int> arbol;
  int  opcode, aux; cin >> opcode;

  while(opcode != -1)
  {
    cin >> aux;

    try
    {
      if(opcode == 0)     {arbol.print(); cout << endl;}
      else if(opcode == 1) arbol.find(aux);
      else if(opcode == 2) arbol.insert(aux);
      else if(opcode == 3) arbol.erase(aux);
      else if(opcode == 4) {if(arbol.empty()) cout << "VACIO\n";
                            else           cout << "NO VACIO\n";}
    }
    catch (const char* msg)
    {
      cerr << msg << endl;
    }
    catch (exception &e)
    {
      cout << e.what() << endl;
    }

    cin >> opcode;
  }
}
\end{lstlisting}

Se muestra a continuación el código correspondiente al test automático. 
Realiza primero un número de inserciones de elementos aleatorios, luego una
secuencia de búsquedas de elementos también aleatorios y en el mismo rango que
en el caso anterior y por último una serie de borrados (en el ejemplo 1000).
Para terminar muestra los elementos restantes en el árbol. Si sucede algún 
error durante el test se muestra por consola la excepción recibida.

\begin{lstlisting}
#include "splay_tree.h"
#include <cstdlib>
#include <time.h>
using namespace std;

int main()
{
  splay_tree<int> arbol;
  int upLimit = 1000, nextValue;
  srand(time(NULL));

  try
  {
    for(int i = 0; i < 1000; i++)
    {
      nextValue = rand()%upLimit;
      arbol.insert(nextValue);
    }

    for(int i = 0; i < 1000; i++)
    {
      nextValue = rand()%upLimit;
      arbol.find(nextValue);
    }

    for(int i = 0; i < 4000; i++)
    {
      nextValue = rand()%upLimit;
      arbol.erase(nextValue);
    }
  }
  catch (const char* msg)  {
    cerr << msg << endl;
  }
  catch (exception &e)  {
    cout << e.what() << endl;
  }

  arbol.print(); 
}
\end{lstlisting}


%++++++++++++++++++++++++++++++++++++++++
% References section will be created automatically 
% with inclusion of "thebibliography" environment
% as it shown below. See text starting with line
% \begin{thebibliography}{99}
% Note: with this approach it is YOUR responsibility to put them in order
% of appearance.

\begin{thebibliography}{99}

\bibitem{ApuntesRicardo}
Ricardo Peña, \textit{Árboles de búsqueda autoajustables (Splay trees)}, 
(Apuntes de la asignatura "Métodos algorítmicos en resolución de problemas",
Facultad de informática, Universidad Complutense, Madrid, curso 2015/16).

\bibitem{Horowitz}
E. Horowitz, S. Sahni, and D. Mehta. \textit{Fundamentals of Data 
Structures in C++}. Computer Science Press, 4th edition, 1997. Capítulo 10.7.

\bibitem{Codeforces}
\textit{Splay tree and it's implementation},  available at 
\texttt{https://codeforces.com/blog/entry/18462}.

\bibitem{Wikipedia} 
\textit{Splay tree},  available at 
\texttt{http://en.wikipedia.org/wiki/Splay\_tree}.

\end{thebibliography}


\end{document}
