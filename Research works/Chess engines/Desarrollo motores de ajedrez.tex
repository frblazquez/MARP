%   Copyright © 2019
%
%   Francisco Javier Blázquez Martínez ~ frblazqu@ucm.es
%
%   Double degree in Mathematics-Computer engineering.
%   Complutense university, Madrid.

%--------------------------------------------------------------------------
% CABECERA DEL DOCUMENTO - IMPORTACIÓN DE PAQUETES
%--------------------------------------------------------------------------
\documentclass[letterpaper,12pt]{article}
\usepackage{tabularx} % extra features for tabular environment
\usepackage{amsmath}  % improve math presentation
\usepackage{graphicx} % takes care of graphic including machinery
\usepackage[margin=1in,letterpaper]{geometry} % decreases margins
\usepackage{cite} % takes care of citations
\usepackage[final]{hyperref} % adds hyper links inside the generated pdf file
\hypersetup{
	colorlinks=true,       % false: boxed links; true: colored links
	linkcolor=blue,        % color of internal links
	citecolor=blue,        % color of links to bibliography
	filecolor=magenta,     % color of file links
	urlcolor=black         
}

\usepackage[utf8]{inputenc}                 % Using Unicode encoding
\usepackage[spanish]{babel}                 % Writing in Spanish language
%\usepackage{listings}                       % For including C++ code
\setlength{\parindent}{0mm}                 % Paragraph indentation null
\setlength{\parskip}{\baselineskip}         % Distance among paragraphs

%--------------------------------------------------------------------------
%   PARA INCLUIR CÓDIGO C++ EN EL DOCUMENTO
%--------------------------------------------------------------------------
\usepackage{color}
\definecolor{gray97}{gray}{.97}
\definecolor{gray75}{gray}{.75}
\definecolor{gray45}{gray}{.45}

\usepackage{listings}
\lstset{ frame=Ltb,
framerule=0pt,
aboveskip=0.5cm,
framextopmargin=3pt,
framexbottommargin=3pt,
framexleftmargin=0.4cm,
framesep=0pt,
rulesep=.4pt,
backgroundcolor=\color{gray97},
rulesepcolor=\color{black},
%
stringstyle=\ttfamily,
showstringspaces = false,
basicstyle=\small\ttfamily,
commentstyle=\color{gray45},
keywordstyle=\bfseries,
%
numbers=left,
numbersep=15pt,
numberstyle=\tiny,
numberfirstline = false,
breaklines=true,
}

% minimizar fragmentado de listados
\lstnewenvironment{listing}[1][]
{\lstset{#1}\pagebreak[0]}{\pagebreak[0]}

\lstdefinestyle{consola}
{basicstyle=\scriptsize\bf\ttfamily,
backgroundcolor=\color{gray75},
}

\lstdefinestyle{C}
{language=C,
}

%--------------------------------------------------------------------------
% INICIO DEL DOCUMENTO - TÍTULO, AUTOR, FECHA ...
%--------------------------------------------------------------------------
\begin{document}

\title{\bf{Breve introducción al desarrollo de motores de ajedrez}}
\author{Francisco Javier Blázquez Martínez}
\date{}
\maketitle

%--------------------------------------------------------------------------
% RESUMEN INICIAL
%--------------------------------------------------------------------------
\begin{abstract}

Trabajo realizado en el marco de la asignatura \textit{Métodos algorítmicos 
en resolución de problemas}. Se introducen diversas técnicas para el desarrollo
de motores de ajedrez centradas principalmente en los algoritmos de búsqueda
y exploración en el árbol de posibles movimientos. He optado por omitir en su 
mayor parte los detalles técnicos de representación del tablero, movimientos y 
reglas del juego por mayor simplicidad.

Se incluye también un apéndice con la aplicación práctica de parte de estas técnicas
presentadas a un problema planteado en clase, el problema de la mochila múltiple.

\end{abstract}

%--------------------------------------------------------------------------
%INICIO DEL CUERPO DEL DOCUMENTO
%--------------------------------------------------------------------------
\section{Introducción}
Hola caracola, en esto he usado \cite{CPW}



%\cite{melissinos, Cyr, Wiki}   % Para tener un ejemplo de cómo citar

\section{Implementación}

\section{Algoritmos}
\subsection{Atributos de la clase}
\subsection{Inserción}
\subsection{Borrado}
\subsection{Rotación de un nodo}
\subsection{Splay}

\appendix
\section{Tests}
Por mayor sencillez a la hora de comprobar el correcto funcionamiento de la 
clase he incluido dos tests, uno manual y otro automático. Esta sección tiene
el objetivo de explicar el funcionamiento de estos para que puedan ser 
ejecutados por cualquier persona con interés en apreciar los múltiples cambios
de estructura de los splay trees en una secuencia de operaciones. Se puede
también fácilmente incluir mediciones del tiempo de ejecución en el test 
automático para convencerse de que el coste amortizado de las operaciones es 
logarítmico.


\begin{lstlisting}
#include "splay_tree.h"
#include <cstdlib>
#include <time.h>
using namespace std;

int main()
{

}
\end{lstlisting}


%++++++++++++++++++++++++++++++++++++++++
% References section will be created automatically 
% with inclusion of "thebibliography" environment
% as it shown below. See text starting with line
% \begin{thebibliography}{99}
% Note: with this approach it is YOUR responsibility to put them in order
% of appearance.

\begin{thebibliography}{99}

\bibitem{CPW}
\textit{Chess Programming Wiki}. Versión del 15 de abril de 2019. Disponible en: \\
\url{https://www.chessprogramming.org/Main_Page}



\end{thebibliography}


\end{document}
