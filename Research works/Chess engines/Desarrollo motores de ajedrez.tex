%   Copyright © 2019
%
%   Francisco Javier Blázquez Martínez ~ frblazqu@ucm.es
%
%   Double degree in Mathematics-Computer engineering.
%   Complutense university, Madrid.

%--------------------------------------------------------------------------
% CABECERA DEL DOCUMENTO - IMPORTACIÓN DE PAQUETES
%--------------------------------------------------------------------------
\documentclass[letterpaper,12pt]{article}
\usepackage{tabularx} % extra features for tabular environment
\usepackage{amsmath}  % improve math presentation
\usepackage{graphicx} % takes care of graphic including machinery
\usepackage[margin=1in,letterpaper]{geometry} % decreases margins
\usepackage{cite} % takes care of citations
\usepackage[final]{hyperref} % adds hyper links inside the generated pdf file
\hypersetup{
	colorlinks=true,       % false: boxed links; true: colored links
	linkcolor=blue,        % color of internal links
	citecolor=blue,        % color of links to bibliography
	filecolor=magenta,     % color of file links
	urlcolor=black         
}

\usepackage[utf8]{inputenc}                 % Using Unicode encoding
\usepackage[spanish]{babel}                 % Writing in Spanish language
%\usepackage{listings}                       % For including C++ code
\setlength{\parindent}{0mm}                 % Paragraph indentation null
\setlength{\parskip}{\baselineskip}         % Distance among paragraphs

%--------------------------------------------------------------------------
%   PARA INCLUIR CÓDIGO C++ EN EL DOCUMENTO
%--------------------------------------------------------------------------
\usepackage{color}
\definecolor{gray97}{gray}{.97}
\definecolor{gray75}{gray}{.75}
\definecolor{gray45}{gray}{.45}

\usepackage{listings}
\lstset{ frame=Ltb,
framerule=0pt,
aboveskip=0.5cm,
framextopmargin=3pt,
framexbottommargin=3pt,
framexleftmargin=0.4cm,
framesep=0pt,
rulesep=.4pt,
backgroundcolor=\color{gray97},
rulesepcolor=\color{black},
%
stringstyle=\ttfamily,
showstringspaces = false,
basicstyle=\small\ttfamily,
commentstyle=\color{gray45},
keywordstyle=\bfseries,
%
numbers=left,
numbersep=15pt,
numberstyle=\tiny,
numberfirstline = false,
breaklines=true,
}

% minimizar fragmentado de listados
\lstnewenvironment{listing}[1][]
{\lstset{#1}\pagebreak[0]}{\pagebreak[0]}

\lstdefinestyle{consola}
{basicstyle=\scriptsize\bf\ttfamily,
backgroundcolor=\color{gray75},
}

\lstdefinestyle{C}
{language=C,
}

%--------------------------------------------------------------------------
%   PARA INCLUIR TABLEROS DE AJEDREZ ESCRITOS
%--------------------------------------------------------------------------
\usepackage{xskak}
\usepackage{chessboard}

\usepackage{chessfss}

\newlength{\symsize}\setlength{\symsize}{24pt}
\newlength{\boardwidth}\setlength{\boardwidth}{8\symsize}
\setboardfontsize{\symsize}

\newcommand{\board}[1]{\setlength{\fboxsep}{0pt}%
\fbox{\parbox{\boardwidth}{\setlength{\baselineskip}{\symsize}#1}}}
\newcommand{\row}[1]{\parbox[c][\symsize]{\symsize}{\hfill{#1}}}
\newcommand{\col}[1]{\parbox[b]{\symsize}{\hfil{#1}}}
\newcommand{\chessboardB}[1]{\begin{tabular}{cc}
\parbox{\symsize}{\setlength{\baselineskip}{\symsize}
\row{8} \row{7} \row{6} \row{5} \row{4} \row{3} \row{2} \row{1}}&\board{#1}\\
\row{ } &\mbox{\col{a}\col{b}\col{c}\col{d}\col{e}\col{f}\col{g}\col{h}}
\end{tabular}}

\newcommand{\TextOnWhite}[1]{\WhiteEmptySquare\hspace{-\symsize}%
\raisebox{.35\symsize}{\makebox[\symsize][c]{\small #1}}}
\newcommand{\TextOnBlack}[1]{\BlackEmptySquare\hspace{-\symsize}%
\raisebox{.35\symsize}{\makebox[\symsize][c]{\small #1}}}

%--------------------------------------------------------------------------
% INICIO DEL DOCUMENTO - TÍTULO, AUTOR, FECHA ...
%--------------------------------------------------------------------------
\begin{document}

\title{\bf{Breve introducción al desarrollo de motores de ajedrez}}
\author{Francisco Javier Blázquez Martínez}
\date{}
\maketitle

%--------------------------------------------------------------------------
% RESUMEN INICIAL
%--------------------------------------------------------------------------
\begin{abstract}

Trabajo realizado en el marco de la asignatura \textit{Métodos algorítmicos 
en resolución de problemas}. Se introducen diversas técnicas para el desarrollo
de motores de ajedrez centradas principalmente en los algoritmos de búsqueda
y exploración en el árbol de posibles movimientos. He optado por omitir en su 
mayor parte los detalles técnicos de representación del tablero, movimientos y 
reglas del juego por mayor simplicidad.

Se incluye también un apéndice con la aplicación práctica de parte de estas técnicas
presentadas a un problema planteado en clase, el problema de la mochila múltiple.

\end{abstract}

%--------------------------------------------------------------------------
%INICIO DEL CUERPO DEL DOCUMENTO
%--------------------------------------------------------------------------
\section{Introducción}

El ajedrez es sin lugar a dudas uno de los juegos más extendidos y con mayor 
impacto en nuestra sociedad. Es de sobra conocido por todos su alta complejidad 
y la dificultad de alcanzar un buen nivel de juego en este. Esto, junto con toda
la historia que acarrea este deporte así como con la concepción social del juego,
muchas veces tan relacionado con la inteligencia, concentración y el desarrollo 
cognitivo lo ha hecho objeto de estudio tantas veces hasta nuestros tiempos.

Es por tanto lógico que, con el auge de la computación y la aparición de nuevas
ramas del conocimiento en este ámbito (como la \textit{inteligencia artificial})
se tratara de aplicar los nuevos conocimientos y nuevas técnicas a la resolución
de este juego. El término ``resolución'' ha sido empleado a drede pues, en contra
de lo que mucha gente pueda pensar, no se conoce (tampoco por los motores de juego
más potentes) una estrategia de juego perfecto, esto es, que garantice el mejor
resultado posible partiendo de una posición dada cualquiera (resolución fuerte 
en el sentido de \cite{Checkers is solved}) ni tampoco una estrategia que garantice
la victoria para algún bando o las tablas a partir de la posición inicial (resolución
débil en el sentido de \cite{Checkers is solved}). Es más, también se desconoce 
si la aparente ventaja de las blancas en la posición incial permitiría que, con
juego perfecto (independientemente de conocer la estrategia) se pudiera forzar 
siempre una victoria o un empate. 

El hecho del que surge esta confusión, pensar que el ajedrez posee una estrategia 
de juego óptima que puede ser calculada por los ordenadores modernos, es la abismal
superioridad con respecto al juego humano de estos en la actualidad. Es más, el
desarrollo de motores de juego para el ajedrez en computadoras, en un principio 
antagonista al juego humano y rechazado por gran parte de la élite del ajedrez 
mundial, rápidamente con la mejora de los ordenadores personales se confirmó como 
la más útil herramienta para el aprendizaje y el estudio del ajedrez. Actualmente,
los grandes jugadores cuentan como herramienta imprescindible de estudio y mejora
de su juego con unas grandes bases de datos de aperturas (primeros movimientos de
cada partida) y un potente motor de ajedrez que les permite analizar posiciones
concretas y analizar sus errores en partidas ya disputadas. Esto se opone a las 
legiones de analistas que antiguamente rodeaban a los jugadores de primera fila
mundial. Gracias a esto, el ajedrez humano de alto nivel se ha ``democratizado'',
siendo posible mejorar enormemente en breves periodos de tiempo a cualquier jugador
con buena capacidad de juego y motivación con la única ayuda de un sencillo motor
de análisis. Esto se puede apreciar en el gran aumento del numero de jugadores con
un ELO \cite{Intro5} mayor de 2000 puntos.


Es el propósito de este breve artículo introducir algunas técnicas 
(principalmente algorítmicas) que, junto con la gran capacidad de cómputo de los
procesadores actuales hacen que los mejores jugadores de ajedrez del mundo 
actualmente no sean de carne y hueso. La mayoría de estas técnicas aparecen en
\cite{CPW} aunque de forma poco profunda  (especialmente en cuanto a los esquemas
algorítmicos se refiere). Esta ha sido la fuente principal de información de este
artículo a partir de la cual otras muchasreferencias han completado los puntos 
que no quedaban totalmente claros o eran en exceso ambiguos. Para una introducción
más profunda al terreno de los motores de ajedrez y su historia, desde sus inicios
hasta la actualidad, donde el motor de Google AlphaZero desarrollado con técnicas
distintas a los motores de ajedrez clásicos (basadas en inteligencia artificial y
redes neuronales) venció al considerado mejor motor de ajedrez, Stockfish 10, 
pasando por el conocido caso de Deep Blue y la primera victoria a un vigente 
campeón del mundo recomiendo leer \cite{Intro1, Intro2, Intro3, Intro4}.


\section{Fundamentos del desarrollo de motores de juego}

Los motores de ajedrez requieren irremediablemente de una serie de componentes para
funcionar. Tratamos aquí de introducirlos dando así al mismo tiempo una visión más 
global de la estructura de este documento.

El primero de ellos, como no podía ser de otra forma, es una representación
del tablero y de las piezas, más aún que esto, una representación del estado del 
tablero. Hace falta una representación del tablero completa más alla del tablero y la
colocación de sus piezas pues, una misma situación del tablero puede darse con el turno
de movimiento para las blancas o para las negras. Otros factores que considerar son los
derechos de enroque de ambos reyes (pueden ser distintos en posiciones con idéntica
distribución) y el número de movimientos reversibles consecutivos por parte de ambos 
oponentes (pues llegar a 50 de estos movimientos consecutivos implica las tablas). Esta
es la parte más técnica del desarrollo del motor de juego, requiere tomar una serie de
decisiones de desarrollo (principalmente las estructuras de datos que subyacen a esta
representación) que son cruciales para posteriormente la ejecución de los movimientos y
la exploración del árbol de juego.

El segundo de estos componentes es un mecanismo para detectar y poder ejecutar (de la 
forma más rápida posible, este punto será crítico para el rendimiento) los movimientos
legales a partir de una posición dada, esto es, de su representación interna. Vemos así
claramente que hay una dependencia absoluta con la representación del tablero elegida.

Partiendo de esta base, que constituye el cuerpo de todo motor de ajedrez (una
representación interna que permite la ejecución sobre ella de cualquier partida de ajedrez
con las reglas de juego actuales) tenemos que construir un sistema ``inteligente'', que
sepa por sí mismo decidir que movimiento ejecutar ante cada posición del tablero. Surge 
así la necesidad de incorporar dos nuevos elementos a nuestro motor de ajedrez. Un sistema
de evaluación que nos permita intuir qué posiciones son favorables para nosotros y qué 
posiciones son favorables para nuestro adversario y, teniendo esto, un mecanismo para 
explorar las posibles posiciones causadas por un movimiento. Esto último, un mecanismo de 
búsqueda en el árbol de posibles movimientos.

Entraremos ahora más en detalle en la representación del tablero, movimientos, evaluación
de las posiciones y especialmente en algoritmos de búsqueda. Todas las fuentes consultadas
pueden verse en la bibliografía, pero he optado por destacar en este punto las empleadas en
esta breve introducción para desarrolladores \cite{Fundamentos2, Fundamentos1}. Una gran 
fuente de ejemplos concretos de motores de ajedrez desarrollados y de código abierto puede
consultarse en \cite{Open Source Chess Engines}.


\section{Representación del tablero}

Como ya hemos dicho, las decisiones de implementación que tomemos en este punto son 
cruciales para el rendimiento final de nuestro motor de juego. De estas depende
directamente el coste de analizar los movimientos posibles y ejecutarlos, la base de
nuestro sistema de búsqueda del mejor movimiento. Otro factor a tener en cuenta es la
cantidad de memoria de la que puede disponer nuestro motor durante la ejecución. Algunas 
estructuras presentadas fueron ampliamente usadas en la segunda mitad del siglo XX y
posteriormente cayeron paulatinamente en desuso con el aumento de la capacidad de las 
memorias. Analizamos ahora una serie de estructuras de datos ideadas para la 
representación de un tablero de ajedrez. Para una información más detallada recomiendo
consultar \cite{BoardRepresentation1, BoardRepresentation2, BoardRepresentation3}.

\begin{itemize}
    \item \textbf{Tablero 8x8:} \\
    Primera aproximación surgida, consistente en la representación de la totalidad del
    tablero, las 64 casillas, indicando en cada una la presencia o ausencia de una pieza,
    el tipo de esta y su color. Generalmente se implementa con un array unidimensional
    de longitud 64 indexado como se ve en la figura.
    
%    \begin{center}
%        \chessboard[showmover=false]
%    \end{center}
    
    \begin{center}\chessboardB{
    \TextOnWhite{56}\TextOnBlack{57}\TextOnWhite{58}\TextOnBlack{59}%
    \TextOnWhite{60}\TextOnBlack{61}\TextOnWhite{62}\TextOnBlack{63}\\
    \TextOnBlack{48}\TextOnWhite{49}\TextOnBlack{50}\TextOnWhite{51}%
    \TextOnBlack{52}\TextOnWhite{53}\TextOnBlack{54}\TextOnWhite{55} \\
    \TextOnWhite{40}\TextOnBlack{41}\TextOnWhite{42}\TextOnBlack{43}%
    \TextOnWhite{44}\TextOnBlack{45}\TextOnWhite{46}\TextOnBlack{47}\\
    \TextOnBlack{32}\TextOnWhite{33}\TextOnBlack{34}\TextOnWhite{35}%
    \TextOnBlack{36}\TextOnWhite{37}\TextOnBlack{38}\TextOnWhite{39} \\
    \TextOnWhite{24}\TextOnBlack{25}\TextOnWhite{26}\TextOnBlack{27}%
    \TextOnWhite{28}\TextOnBlack{29}\TextOnWhite{30}\TextOnBlack{31}\\
    \TextOnBlack{16}\TextOnWhite{17}\TextOnBlack{18}\TextOnWhite{19}%
    \TextOnBlack{20}\TextOnWhite{21}\TextOnBlack{22}\TextOnWhite{23} \\
    \TextOnWhite{8}\TextOnBlack{9}\TextOnWhite{10}\TextOnBlack{11}%
    \TextOnWhite{12}\TextOnBlack{13}\TextOnWhite{14}\TextOnBlack{15}\\
    \TextOnBlack{0}\TextOnWhite{1}\TextOnBlack{2}\TextOnWhite{3}%
    \TextOnBlack{4}\TextOnWhite{5}\TextOnBlack{6}\TextOnWhite{7}
    }\end{center}
    
    Así, se puede conseguir una representación completa del tablero con 64B (un byte
    por casilla es suficiente para todas las posibilidades), necesitando además de esto
    únicamente almacenar derechos de enroque, derechos de comer al paso y turno de juego.
    Es fácil ver, sin embargo, que buscar todos los movimientos posibles para un jugador
    requiere recorrer todo el tablero y, para cada pieza de este jugador, explorar todas
    las posibles casillas de destino (viendo si están ocupadas por una pieza propia).
    Además, surgieron problemas con esta representación en la detección de movimientos
    que podían ``salir'' del tablero, movimientos a índices fuera de los contemplados.

    \item \textbf{Tablero 10x12:}\\
    Es fácil ver que con la indexación dada a las casillas del tablero en la estructura
    de datos del apartado anterior, moverse a la casilla superior es sumar ocho al 
    índice, a la inferior restar ocho, a la casilla de la derecha es sumar uno (¡no 
    siempre!) y a la de la izquierda es restar uno (¡no siempre!). Sin embargo, como se
    ve fácilmente, en los bordes del tablero surgen situaciones anómalas como sobrepasar
    el rango permitido del índice o hacer ``saltos de tablero'' de la columna derecha a
    la izquierda. Esto hacía muy costosas la comprobación de la legalidad de los
    movimientos generados y, por tanto, aumentaba el coste de la generación de 
    movimientos, punto crítico para el rendimiento del motor. Se ideó entonces esta 
    estructura que engloba el tablero con un margen de casillas marcadas como ilegales.
    Son necesarias las dos filas completas superiores e inferiores debido al movimiento
    de los caballos, que saltan siempre a casillas separadas por una fila completa a su
    posición original.
    
    De esta forma se reducen los costes de comprobación de la legalidad de un movimiento
    y se mantienen los mismos costes de búsqueda de piezas, ejecución de movimiento... 
    Sin embargo, juegue el bando que juege, saber sus posibles movimientos implica buscar
    por todo el tablero sus piezas (técnica que se puede mejorar con un procesado por cada
    movimiento y respuesta o bien añadiendo estructuras de datos adicionales). Esto hizo
    que esta estructura cayera en desuso. La imagen de abajo muestra un esquema de la 
    representación en tableros 10x12.
    
    \begin{center}
         \chessboard
         [
            maxfield=j12,
            startfen=b10,
            addfen=rnbqkbnr/pppppppp/8/8/8/8/PPPPPPPP/RBNQKBNR,
            emphstyle=\color{red},
            emphareas={a12-j11, a1-j2, a3-a10, j3-j10},
            labelleftwidth=1.5ex,
            showmover=false
        ]   
    \end{center}

    \item \textbf{Lista de piezas:} \\
    Frente a las dos representaciónes anteriores, centradas en la representación del 
    tablero completo, surge la idea de que representar las casillas vacías es innecesario
    y complica y aumenta el coste de las búsquedas. De esta forma diversos motores optan
    por la representación del tablero como listas de piezas (blancas y negras) con su
    posición asociada. Así, saber si determinada casilla está libre u ocupada tiene un
    coste mayor (pasa de ser constante a ser del orden del número de piezas que quedan
    en el tablero) pero, a cambio, el coste de la generación de los movimientos legales a
    partir de una determinada posición es menor. En cualquier caso, esta estructura de
    datos simplifica la representación del tablero pero complica también la implementación
    de búsqueda de movimientos, lo que hace que en la práctica la mejora del rendimiento
    sea nula.
    
    \item \textbf{Bitboards:} \\
    Con la aparición de los procesadores de 64 bits, surge una idea muy interesante para
    aplicar al desarrollo de los motores de ajedrez (así como de damas o cualquier juego 
    sobre un tablero de 64 casillas). La idea es que, en una única palabra de 64 bits, 
    puedo reflejar una determinada condición en todas las casillas del tablero. Esto 
    permite que gran parte de las comprobaciones se reduzcan a operaciones lógicas 
    soportadas por el procesador y, por tanto, ejecutables en un número muy bajo de ciclos.
    
    Por ejemplo, podemos tener en una sola palabra de memoria la situación de todos los
    peones blancos (1 en la posición i-ésima implica que hay un peón blanco en esa 
    casilla). Es más, esta idea no vale únicamente para la representación de piezas, sino
    también para considerar casillas de ataque y procesarlas rápidamente. Si tuvieramos
    las casillas a las que pueden saltar los caballos negros, para saber qué peones blancos
    se pueden comer los caballos negros simplemente tendríamos que hacer la AND lógica
    de los dos bitboards introducidos.
    
\end{itemize}
    
Existen otras muchas estructuras para la representación del estado del tablero de juego,
sin embargo, muchas han caído en desuso. Actualmente las más usadas son los bitboards así
como soluciones híbridas, que representan tanto el tablero entero como bitboards para las
piezas individuales reduciendo al máximo los tiempos de comprobaciones a cambio de pagar
los costes en tiempo de las actualizaciones de ambas representaciones.
    

\section{Generación de movimientos}


\section{Evaluación de la posición}


\section{Elección del movimiento}


\appendix
\section{Tests}
Por mayor sencillez a la hora de comprobar el correcto funcionamiento de la 
clase he incluido dos tests, uno manual y otro automático. Esta sección tiene
el objetivo de explicar el funcionamiento de estos para que puedan ser 
ejecutados por cualquier persona con interés en apreciar los múltiples cambios
de estructura de los splay trees en una secuencia de operaciones. Se puede
también fácilmente incluir mediciones del tiempo de ejecución en el test 
automático para convencerse de que el coste amortizado de las operaciones es 
logarítmico.


\begin{lstlisting}
#include "splay_tree.h"
#include <cstdlib>
#include <time.h>
using namespace std;

int main()
{

}
\end{lstlisting}


%++++++++++++++++++++++++++++++++++++++++
% References section will be created automatically 
% with inclusion of "thebibliography" environment
% as it shown below. See text starting with line
% \begin{thebibliography}{99}
% Note: with this approach it is YOUR responsibility to put them in order
% of appearance.

\begin{thebibliography}{99}

% General, la joya de la corona de la bibliografía:
\bibitem{CPW}
\textit{Chess Programming Wiki}. Versión del 15 de abril de 2019. Disponible en: \\
\url{https://www.chessprogramming.org/Main_Page}

% Otras dos joyas de la revista science:
\bibitem{Checkers is solved}
Schaeffer, Burch, Björnsson, Kishimoto Müller, Lake, Lu, Sutphen. (2007). ``Checkers is 
solved''. \textit{Revista Science} 317, (5844), 1518-1522. Artículo disponible en: \\
\url{https://science.sciencemag.org/content/317/5844/1518/tab-pdf}

\bibitem{AlphaZero}
Silver, Hubert, Schrittwieser, Antonoglou , Lai, Guez, Lanctot , Sifre, Kumaran, 
Graepel, Lillicrap, Simonyan, Hassabis. (2018). ``A general reinforcement learning 
algorithm that masters chess, shogi, and Go through self-play''. \textit{Revista 
Science} 362 (6419), 1140-1144. Artículo disponible en: \\
\url{https://science.sciencemag.org/content/362/6419/1140/tab-pdf}

% Una fuente de información perfecta
\bibitem{Recommended Reading}
\textit{Lecturas recomendadas}. Versión del 16 de abril de 2019. Disponible en: \\
\url{https://www.chessprogramming.org/Recommended_Reading}

% Historia y bibliografía poco importante para la introducción:
\bibitem{Intro1}
\textit{Computer chess}. Versión del 17 de abril de 2019. Disponible en: \\
\url{https://en.wikipedia.org/wiki/Computer_chess}

\bibitem{Intro2}
\textit{History of computer chess}. Versión del 17 de abril de 2019. Disponible en: \\
\url{https://www.chessprogramming.org/History}

\bibitem{Intro3}
\textit{Deep Blue}. Versión del 17 de abril de 2019. Disponible en: \\
\url{https://en.wikipedia.org/wiki/Deep_Blue_(chess_computer)} \\
\url{https://en.wikipedia.org/wiki/Deep_Blue_versus_Garry_Kasparov}

\bibitem{Intro4}
\textit{AlphaZero}. Versión del 17 de abril de 2019. Disponible en: \\
\url{https://en.wikipedia.org/wiki/AlphaZero} \\
\url{https://en.wikipedia.org/wiki/Leela_Chess_Zero}

\bibitem{Intro5}
\textit{Sistema de puntuación ELO}. Versión del 17 de abril de 2019. Disponible en: \\
\url{https://es.wikipedia.org/wiki/Sistema_de_puntuacion_Elo}

% Introducciones al desarrollo:
\bibitem{Fundamentos1}
\textit{Getting Started in chess programming}. Versión del 18 de abril de 2019. 
Disponible en: \\
\url{https://www.chessprogramming.org/Getting_Started}

\bibitem{Fundamentos2}
\textit{Chess Programming Part I: Getting Started}. Versión del 18 de abril de 2019. 
Disponible: \\
\url{http://archive.gamedev.net/archive/reference/articles/article1014.html}

% Representación del tablero:
\bibitem{BoardRepresentation1}
\textit{Board representation}. Versión del 17 de abril de 2019. Disponible en: \\
\url{https://www.chessprogramming.org/Board_Representation}

\bibitem{BoardRepresentation2}
\textit{Chess Programming Part II: Data Structures}. Versión 18 de abril de 2019. 
Disponible: \\
\url{https://www.gamedev.net/articles/programming/artificial-intelligence/chess-programming-part-ii-data-structures-r1046}

\bibitem{BoardRepresentation3}
\textit{Board representation (chess)}. Versión 18 de abril de 2019. Disponible en: \\
\url{https://en.wikipedia.org/wiki/Board_representation_(chess)#Huffman_encodings}

% Generación de movimientos:
\bibitem{MoveGeneration1}
\textit{Move generation}. Versión del 17 de abril de 2019. Disponible en: \\
\url{https://www.chessprogramming.org/Move_Generation}

\bibitem{MoveGeneration2}
\textit{Chess Programming Part III: Move Generation}. Versión del 17 de abril de 2019.
\\ \url{https://www.gamedev.net/articles/programming/artificial-intelligence/chess-programming-part-iii-move-generation-r1126}

% Motores de juego:
\bibitem{Open Source Chess Engines}
\textit{Open source chess engines}. Versión del 17 de abril de 2019. Disponible en: \\
\url{https://www.chessprogramming.org/Category:Open_Source}

\bibitem{Stockfish}
\textit{Stockfish}. Versión del 18 de abril de 2019. Disponible en: \\
\url{https://www.chessprogramming.org/Stockfish}

\end{thebibliography}


\end{document}

