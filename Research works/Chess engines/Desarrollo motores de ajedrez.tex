%   Copyright © 2019
%
%   Francisco Javier Blázquez Martínez ~ frblazqu@ucm.es
%
%   Double degree in Mathematics-Computer engineering.
%   Complutense university, Madrid.

%--------------------------------------------------------------------------
% CABECERA DEL DOCUMENTO - IMPORTACIÓN DE PAQUETES
%--------------------------------------------------------------------------
\documentclass[letterpaper,12pt]{article}
\usepackage{tabularx} % extra features for tabular environment
\usepackage{amsmath}  % improve math presentation
\usepackage{graphicx} % takes care of graphic including machinery
\usepackage[margin=1in,letterpaper]{geometry} % decreases margins
\usepackage{cite} % takes care of citations
\usepackage[final]{hyperref} % adds hyper links inside the generated pdf file
\hypersetup{
	colorlinks=true,       % false: boxed links; true: colored links
	linkcolor=blue,        % color of internal links
	citecolor=blue,        % color of links to bibliography
	filecolor=magenta,     % color of file links
	urlcolor=black         
}

\usepackage[utf8]{inputenc}                 % Using Unicode encoding
\usepackage[spanish]{babel}                 % Writing in Spanish language
%\usepackage{listings}                       % For including C++ code
\setlength{\parindent}{0mm}                 % Paragraph indentation null
\setlength{\parskip}{\baselineskip}         % Distance among paragraphs

%--------------------------------------------------------------------------
%   PARA INCLUIR CÓDIGO C++ EN EL DOCUMENTO
%--------------------------------------------------------------------------
\usepackage{color}
\definecolor{gray97}{gray}{.97}
\definecolor{gray75}{gray}{.75}
\definecolor{gray45}{gray}{.45}

\usepackage{listings}
\lstset{ frame=Ltb,
framerule=0pt,
aboveskip=0.5cm,
framextopmargin=3pt,
framexbottommargin=3pt,
framexleftmargin=0.4cm,
framesep=0pt,
rulesep=.4pt,
backgroundcolor=\color{gray97},
rulesepcolor=\color{black},
%
stringstyle=\ttfamily,
showstringspaces = false,
basicstyle=\small\ttfamily,
commentstyle=\color{gray45},
keywordstyle=\bfseries,
%
numbers=left,
numbersep=15pt,
numberstyle=\tiny,
numberfirstline = false,
breaklines=true,
}

% minimizar fragmentado de listados
\lstnewenvironment{listing}[1][]
{\lstset{#1}\pagebreak[0]}{\pagebreak[0]}

\lstdefinestyle{consola}
{basicstyle=\scriptsize\bf\ttfamily,
backgroundcolor=\color{gray75},
}

\lstdefinestyle{C}
{language=C,
}

%--------------------------------------------------------------------------
% INICIO DEL DOCUMENTO - TÍTULO, AUTOR, FECHA ...
%--------------------------------------------------------------------------
\begin{document}

\title{\bf{Breve introducción al desarrollo de motores de ajedrez}}
\author{Francisco Javier Blázquez Martínez}
\date{}
\maketitle

%--------------------------------------------------------------------------
% RESUMEN INICIAL
%--------------------------------------------------------------------------
\begin{abstract}

Trabajo realizado en el marco de la asignatura \textit{Métodos algorítmicos 
en resolución de problemas}. Se introducen diversas técnicas para el desarrollo
de motores de ajedrez centradas principalmente en los algoritmos de búsqueda
y exploración en el árbol de posibles movimientos. He optado por omitir en su 
mayor parte los detalles técnicos de representación del tablero, movimientos y 
reglas del juego por mayor simplicidad.

Se incluye también un apéndice con la aplicación práctica de parte de estas técnicas
presentadas a un problema planteado en clase, el problema de la mochila múltiple.

\end{abstract}

%--------------------------------------------------------------------------
%INICIO DEL CUERPO DEL DOCUMENTO
%--------------------------------------------------------------------------
\section{Introducción}

El ajedrez es sin lugar a dudas uno de los juegos más extendidos y con mayor 
impacto en nuestra sociedad. Es de sobra conocido por todos su alta complejidad 
y la dificultad de alcanzar un buen nivel de juego en este. Esto, junto con toda
la historia que acarrea este deporte así como con la concepción social del juego,
muchas veces tan relacionado con la inteligencia, concentración y el desarrollo 
cognitivo lo ha hecho objeto de estudio tantas veces hasta nuestros tiempos.

Es por tanto lógico que, con el auge de la computación y la aparición de nuevas
ramas del conocimiento en este ámbito (como la \textit{inteligencia artificial})
se tratara de aplicar los nuevos conocimientos y nuevas técnicas a la resolución
de este juego. El término ``resolución'' ha sido empleado a drede pues, en contra
de lo que mucha gente pueda pensar, no se conoce (tampoco por los motores de juego
más potentes) una estrategia de juego perfecto, esto es, que garantice el mejor
resultado posible partiendo de una posición dada cualquiera (resolución fuerte 
en el sentido de \cite{Checkers is solved}) ni tampoco una estrategia que garantice
la victoria para algún bando o las tablas a partir de la posición inicial (resolución
débil en el sentido de \cite{Checkers is solved}). Es más, también se desconoce 
si la aparente ventaja de las blancas en la posición incial permitiría que, con
juego perfecto (independientemente de conocer la estrategia) se pudiera forzar 
siempre una victoria o un empate. 

El hecho del que surge esta confusión, pensar que el ajedrez posee una estrategia 
de juego óptima que puede ser calculada por los ordenadores modernos, es la abismal
superioridad con respecto al juego humano de estos en la actualidad. Es más, el
desarrollo de motores de juego para el ajedrez en computadoras, en un principio 
antagonista al juego humano y rechazado por gran parte de la élite del ajedrez 
mundial, rápidamente con la mejora de los ordenadores personales se confirmó como 
la más útil herramienta para el aprendizaje y el estudio del ajedrez. Actualmente,
los grandes jugadores cuentan como herramienta imprescindible de estudio y mejora
de su juego con unas grandes bases de datos de aperturas (primeros movimientos de
cada partida) y un potente motor de ajedrez que les permite analizar posiciones
concretas y analizar sus errores en partidas ya disputadas. Esto se opone a las 
legiones de analistas que antiguamente rodeaban a los jugadores de primera fila
mundial. Gracias a esto, el ajedrez humano de alto nivel se ha ``democratizado'',
siendo posible mejorar enormemente en breves periodos de tiempo a cualquier jugador
con buena capacidad de juego y motivación con la única ayuda de un sencillo motor
de análisis. Esto se puede apreciar en el gran aumento del numero de jugadores con
un ELO \cite{Intro5} mayor de 2000 puntos.


Es el propósito de este breve artículo introducir algunas técnicas 
(principalmente algorítmicas) que, junto con la gran capacidad de cómputo de los
procesadores actuales hacen que los mejores jugadores de ajedrez del mundo 
actualmente no sean de carne y hueso. La mayoría de estas técnicas aparecen en
\cite{CPW} aunque de forma poco profunda  (especialmente en cuanto a los esquemas
algorítmicos se refiere). Esta ha sido la fuente principal de información de este
artículo a partir de la cual otras muchasreferencias han completado los puntos 
que no quedaban totalmente claros o eran en exceso ambiguos. Para una introducción
más profunda al terreno de los motores de ajedrez y su historia, desde sus inicios
hasta la actualidad, donde el motor de Google AlphaZero desarrollado con técnicas
distintas a los motores de ajedrez clásicos (basadas en inteligencia artificial y
redes neuronales) venció al considerado mejor motor de ajedrez, Stockfish 10, 
pasando por el conocido caso de Deep Blue y la primera victoria a un vigente 
campeón del mundo recomiendo leer \cite{Intro1, Intro2, Intro3, Intro4}.


\section{Fundamentos del desarrollo de motores de juego}

\section{Algoritmos}
\subsection{Atributos de la clase}
\subsection{Inserción}
\subsection{Borrado}
\subsection{Rotación de un nodo}
\subsection{Splay}

\appendix
\section{Tests}
Por mayor sencillez a la hora de comprobar el correcto funcionamiento de la 
clase he incluido dos tests, uno manual y otro automático. Esta sección tiene
el objetivo de explicar el funcionamiento de estos para que puedan ser 
ejecutados por cualquier persona con interés en apreciar los múltiples cambios
de estructura de los splay trees en una secuencia de operaciones. Se puede
también fácilmente incluir mediciones del tiempo de ejecución en el test 
automático para convencerse de que el coste amortizado de las operaciones es 
logarítmico.


\begin{lstlisting}
#include "splay_tree.h"
#include <cstdlib>
#include <time.h>
using namespace std;

int main()
{

}
\end{lstlisting}


%++++++++++++++++++++++++++++++++++++++++
% References section will be created automatically 
% with inclusion of "thebibliography" environment
% as it shown below. See text starting with line
% \begin{thebibliography}{99}
% Note: with this approach it is YOUR responsibility to put them in order
% of appearance.

\begin{thebibliography}{99}

% General, la joya de la corona de la bibliografía:
\bibitem{CPW}
\textit{Chess Programming Wiki}. Versión del 15 de abril de 2019. Disponible en: \\
\url{https://www.chessprogramming.org/Main_Page}

% Otras dos joyas de la revista science:
\bibitem{Checkers is solved}
Schaeffer, Burch, Björnsson, Kishimoto Müller, Lake, Lu, Sutphen. (2007). ``Checkers is solved''. \textit{Revista Science} 317, (5844), 1518-1522. Artículo disponible en: \\
\url{https://science.sciencemag.org/content/317/5844/1518/tab-pdf}

\bibitem{AlphaZero}
Silver, Hubert, Schrittwieser, Antonoglou , Lai, Guez, Lanctot , Sifre, Kumaran, 
Graepel, Lillicrap, Simonyan, Hassabis. (2018). ``A general reinforcement learning 
algorithm that masters chess, shogi, and Go through self-play''. \textit{Revista 
Science} 362 (6419), 1140-1144. Artículo disponible en: \\
\url{https://science.sciencemag.org/content/362/6419/1140/tab-pdf}

% Una fuente de información perfecta
\bibitem{Recommended Reading}
\textit{Lecturas recomendadas}. Versión del 16 de abril de 2019. Disponible en: \\
\url{https://www.chessprogramming.org/Recommended_Reading}

% Historia y bibliografía poco importante para la introducción:
\bibitem{Intro1}
\textit{Computer chess}. Versión del 17 de abril de 2019. Disponible en: \\
\url{https://en.wikipedia.org/wiki/Computer_chess}

\bibitem{Intro2}
\textit{History of computer chess}. Versión del 17 de abril de 2019. Disponible en: \\
\url{https://www.chessprogramming.org/History}

\bibitem{Intro3}
\textit{Deep Blue}. Versión del 17 de abril de 2019. Disponible en: \\
\url{https://en.wikipedia.org/wiki/Deep_Blue_(chess_computer)} \\
\url{https://en.wikipedia.org/wiki/Deep_Blue_versus_Garry_Kasparov}

\bibitem{Intro4}
\textit{AlphaZero}. Versión del 17 de abril de 2019. Disponible en: \\
\url{https://en.wikipedia.org/wiki/AlphaZero} \\
\url{https://en.wikipedia.org/wiki/Leela_Chess_Zero}

\bibitem{Intro5}
\textit{Sistema de puntuación ELO}. Versión del 17 de abril de 2019. Disponible en: \\
\url{https://es.wikipedia.org/wiki/Sistema_de_puntuacion_Elo}

% Introducciones al desarrollo:
\bibitem{Fundamentos1}
\textit{Getting Started in chess programming}. Versión del 18 de abril de 2019. 
Disponible en: \\
\url{https://www.chessprogramming.org/Getting_Started}

\bibitem{Fundamentos2}
\textit{Chess Programming Part I: Getting Started}. Versión del 18 de abril de 2019. 
Disponible: \\
\url{http://archive.gamedev.net/archive/reference/articles/article1014.html}


% Motores de juego:
\bibitem{Open Source Chess Engines}
\textit{Open source chess engines}. Versión del 17 de abril de 2019. Disponible en: \\
\url{https://www.chessprogramming.org/Category:Open_Source}

\bibitem{Stockfish}
\textit{Stockfish}. Versión del 18 de abril de 2019. Disponible en: \\
\url{https://www.chessprogramming.org/Stockfish}

\end{thebibliography}


\end{document}

